\chapter{Final conclusions}
\label{conclusions}

\section{Project evaluation}

All of the planned functionalities have been implemented. Application:
\begin{itemize} [noitemsep]
\item displays a level (path, objects, cursor) properly.
\item can correctly detect the progress
\item has fully functioning levelling system.
\item can correctly measure player's score and data
\end{itemize}
The game is ready to be presented to a children with dyspraxia as a rehabilitation tool. Although adding some more children-friendly features (like sounds, more colourful and friendly objects) may be considered to allows better interaction with patients, it is not required for application to work properly.

\section{Similar project results}
Similar application was developed by authors of \cite{13}. Firstly,  children's (with and without DCD) score in performing the task(similar to the one in game, but without magnetic attraction) was checked. Then, children were trained once a week for about 20 minutes with usage of application, up to five times. After this, they were tested again. Before the training, results of children with DCD were significantly worse than results of typically developing children. After the training, both groups improve, but the progress of children with dyspraxia was notably greater. It means that it is possible to overcome the "catch-22" syndrome and children with DCD can actually learn complex motor skills with proper support and training method. 

\section{Summary}
The project was successful. All requirements have been implemented. Fully functional, stand-alone application has been created. The Open Haptics API was a good choice - developing application with this library was convenient and fast. It surely has great potential and can be use to create innovative applications that uses the Phantom Omni manipulator.

The next stage of the project should be testing it with children, so the game fulfils their requirements and than perform tests, where application would be used as a rehabilitation tool for children with dyspraxia.

Development Coordination Disorder is troublesome disease, also because of psychological and social reasons. Engaging in the development of treatmeant for this disorder was really valuable experience.

