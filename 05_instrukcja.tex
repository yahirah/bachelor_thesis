\chapter{User's manual}

This manual describes game requirements and controls. 


\section{Launching prerequisites}
\begin{itemize}
\item In order to run the application it is required to plug Phantom Omni Device (through 6-pin FireWire port) and install a proper, up-to-date driver. 
\item Suggested systems are Microsoft Windows XP/Vista/7 (32/64 bit). 
\item The graphic card should support OpenGL version (at least) 3.2. 
\end{itemize}

\section{General tips}
General clues about playing the game are:
\begin{enumerate} [noitemsep]
\item Round is started by touching the curve. 
\item The precision of a player's movement is more important than it's speed. Moving too fast will cause not acquiring the points player passed(because of low sampling frequency). 
\item Game loads new level(or reset the level) automatically.
\end{enumerate}

\section{Controls}
Three main input devices that control the application are:
\begin{itemize} [noitemsep]
\item \emph{Phantom Omni} - controls the cursor during the game. There are two buttons on the stylus, but they have no influence on application.
\item \emph{mouse} - controls the camera. Moving while pressing left button allows to rotate the camera, right button - to translate, and middle one - to scale the view.
\item \emph{keyboard} - responsible for debug/advanced options, not available in normal game
\end{itemize}
The player should use only Phantom Omni to control the game and eventually the mouse to adjust the view. The rest should be controlled by the supervisor. 

\section{Advanced options}
Advanced options are available only from keyboard. They allow to toggle some elements of the scene or use cheats. 
Possible short-cuts are:
\begin{itemize} [noitemsep]
\item "s" - toggle skybox 
\item "r" - toggle the route view(of player's object)
\item "p" - toggle the control points of Curve view
\item "w" - immediate passage to next level
\item "q" - quit the game
\end{itemize}

\section{Quitting the game}
Quitting the game is possible by closing the window or pressing the "q" key. 
